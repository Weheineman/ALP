\documentclass{article}


\usepackage[utf8x]{inputenc}
\usepackage[spanish]{babel}
\usepackage{amssymb}
\usepackage{proof}
\usepackage{xcolor}
\usepackage{listings}
\usepackage{graphicx}
\usepackage{etoolbox}
\usepackage[most]{tcolorbox}
\usepackage[hidelinks]{hyperref}
\usepackage[nottoc,numbib]{tocbibind}
\hyphenchar\font=-1

\hypersetup{
	allcolors=black
}

\renewcommand{\contentsname}{\'Indice}
\renewcommand\refname{Referencias}


% set the default code style
\lstset{
    language = Haskell,
    frame=tblr, % draw a frame at the top and bottom of the code block
    rulecolor=\color{black},
    tabsize=4, % tab space width
    breakatwhitespace=true,         
    breaklines=true,                 
    captionpos=b,                    
    keepspaces=false,
    showspaces=false,                
    showstringspaces=false,
    showtabs=false,
    basicstyle=\small\ttfamily\bfseries,
    commentstyle=\color{red}, % comment color
    keywordstyle=\color{blue}, % keyword color
    stringstyle=\color{orange} % string color
}
\title{Trabajo Pr\'actico 1}
\date{28-09-2018}
\author{
	Weinand, Gianni Georg\\
	\texttt{W-0528/2}
	\and
	Capretto, Margarita
	\texttt{C-6055/1}
}


\newtcblisting{commandshell}{colback=black,colupper=white,colframe=black,
listing only,listing options={language=sh,  keywordstyle=\color{white}, basicstyle=\small\ttfamily\bfseries, breaklines=true, keepspaces=true, breakatwhitespace=false,  captionpos=b, showstringspaces=false, showtabs=false },
every listing line={\textcolor{pink}{\small\ttfamily\bfseries Monika \$> }}}


\begin{document}

\begin{titlepage}
	\centering
	\includegraphics[width=0.4\textwidth]{unrlogo}\par\vspace{1cm}
	{\scshape\LARGE R-313 An\'alisis de Lenguajes de Programaci\'on \par}
	\vspace{1cm}
	{\scshape\Large Trabajo Pr\'actico 1\par}
	\vspace{1.5cm}
	\vspace{2cm}
	{\Large\itshape Gianni Georg Weinand W-0528/2 \\ Margarita Capretto C-6055/1\par}

	\vfill

% Bottom of the page
	{\large 28 de Septiembre, 2018 \par}
\end{titlepage}

%\tableofcontents

\newpage

\section{Ejercicio 1}

\subsection{Sintaxis Abstracta}
\begin{align*}
intexp ::=\ &nat\ |\ var\ |\ -_{u} intexp\\
|\ &intexp + intexp\\
|\ &intexp -_{b} intexp\\
|\ &intexp \times intexp\\
|\ &intexp \div intexp\\
|\ &boolexp\ ?\ intexp : intexp
\end{align*}

\subsection{Sintaxis Concreta}

\begin{align*}
	intexp ::=\ &nat\\
	|\ &var\\
	|\ &\text{'-'}\ intexp\\
	|\ &intexp\ \text{'+'}\ intexp\\
	|\ &intexp\ \text{'-'}\ intexp\\
	|\ &intexp\ \text{'*'}\ intexp\\
	|\ &intexp\ \text{'/'}\  intexp\\
	|\ &\text{'('}\ intexp\ \text{')'}\\
	|\ &boolexp\ \text{'?'}\ intexp\ \text{':'}\ intexp
\end{align*}


\section{Ejercicio 2}
\lstinputlisting{AST.hs}

\section{Ejercicio 3}
\lstinputlisting{Parser.hs}

\section{Ejercicio 4}
\infer[\mbox{TERN}_1]
{\langle p\ ?\ e_{0}: e_{1}, \sigma \rangle  \Downarrow_{\text{intexp}} n_{0} }
{\langle p,\sigma \rangle \Downarrow_{\text{boolexp}} \mathbf{true}\ \ \ \langle e_{0}, \sigma \rangle \Downarrow_{\text{intexp}} n_{0}}

\vspace{5mm}

\infer[\mbox{TERN}_2]
{\langle p\ ?\ e_{0}: e_{1}, \sigma \rangle  \Downarrow_{\text{intexp}} n_{1} }
{\langle p,\sigma \rangle \Downarrow_{\text{boolexp}} \mathbf{false}\ \ \ \langle e_{1}, \sigma \rangle \Downarrow_{\text{intexp}} n_{1}}

\vspace{5mm}

\section{Ejercicio 5}
\paragraph{}
Sea $\langle t, \sigma \rangle \rightsquigarrow \langle t_1, \sigma_1 \rangle$ y $\langle t, \sigma \rangle \rightsquigarrow \langle t_2, \sigma_2 \rangle$. Se debe probar que $\langle t_1, \sigma_1 \rangle = \langle t_2, \sigma_2 \rangle$.

\paragraph{}
Se demostrar\'a por inducci\'on sobre la derivaci\'on de $\langle t_1, \sigma_1 \rangle$, teniendo en cuenta la \'ultima regla utilizada. 

\paragraph{}
Si es la regla ASS, t es de la forma $v:=e$. Luego
\begin{itemize}
\item
$\langle e, \sigma \rangle \Downarrow_{intexp} n$, para alg\'un $n$.
\item
$t_1 = \mathbf{skip}$ y $\sigma_1 = [ \sigma | v : n ]$.
\item
la \'ultima regla en la derivaci\'on de $\langle t_2, \sigma_2 \rangle$ no pudo ser SEQ$_1$, SEQ$_2$, IF$_1$, IF$_2$ ni REPEAT por la estructura de t. Por lo tanto, la \'ultima regla aplicada debi\'o ser ASS. Entonces, usando que $\Downarrow_{intexp}$ es determinista, se tiene que $t_2 = \mathbf{skip} = t_1$ y $\sigma_2 = [ \sigma | v : n ] = \sigma_1$.
\end{itemize}

\paragraph{}
Si es la regla SEQ$_1$, t es de la forma $\mathbf{skip};c$. Luego
\begin{itemize}
\item
$t_1 = c$ y $\sigma_1 = \sigma$.
\item
la \'ultima regla en la derivaci\'on de $\langle t_2, \sigma_2 \rangle$ no pudo ser ASS, IF$_1$, IF$_2$ ni REPEAT por la estructura de t. Pero tampoco pudo ser SEQ$_2$, ya que no hay una regla de derivaci\'on para $\mathbf{skip}$. Por lo tanto, la \'ultima regla aplicada debi\'o ser tambi\'en SEQ$_1$, entonces $ t_2 = c = t_1$ y $\sigma_2 = \sigma = \sigma_1$.
\end{itemize}

\paragraph{}
Si es la regla SEQ$_2$, t es de la forma $c_0;c_1$. Luego
\begin{itemize}
\item
$t_1 = c_0';c_1 $ y $ \sigma_1 = \sigma'$.
\item
la \'ultima regla en la derivaci\'on de $ \langle t_2, \sigma_2 \rangle $ no pudo ser ASS, IF$_1$, IF$_2$ ni REPEAT por la estructura de t. Por hip\'otesis inductiva se tiene que el paso $\langle c_0, \sigma \rangle \rightsquigarrow \langle c_0',  \sigma' \rangle$ es determinista. Se tiene entonces que $c_0$ no puede ser $ \mathbf{skip}$ y consecuentemente la \'ultima regla en la derivaci\'on de $\langle t_2, \sigma_2 \rangle$ no pudo ser SEQ$_1$. La \'unica posibilidad es que haya sido tambi\'en SEQ$_2$ y se concluye $t_2 = c_0';c_1 = t_1$ y $\sigma_2 = \sigma' = \sigma_1$. 
\end{itemize}

\paragraph{}
Si es la regla IF$_1$, t es de la forma $ \mathbf{if\ } b \mathbf{\ then\ } c_0	 \mathbf{\ else\ } c_1$. Luego
\begin{itemize}
\item
$t_1 = c_0$ y $\sigma_1 = \sigma$.
\item
la \'ultima regla en la derivaci\'on de $\langle t_2, \sigma_2 \rangle$ no pudo ser ASS, SEQ$_1$, SEQ$_2$ ni REPEAT por la estructura de t. Por determinismo de $\Downarrow_{boolexp}$ se tiene que b $\Downarrow_{boolexp} \mathbf{true}$, lo que vuelve imposible el uso de IF$_2$. La \'unica posibilidad es que haya sido tambi\'en IF$_1$ y se concluye $ t_2 = c_0 = t_1$ y $ \sigma_2$ = $ \sigma = \sigma_1 $. 
\end{itemize}

\paragraph{}
Si es la regla IF$_2$, la prueba es an\'aloga a IF$_1$. 

\paragraph{}
Si es la regla REPEAT, t es de la forma $\mathbf{\ repeat\ } c \mathbf{\ until\ } b$. Luego
\begin{itemize}
\item
$ t_1 = c;\mathbf{if\ } b \mathbf{\ then\ skip\ else\ repeat\  } c \mathbf{\ until\ } b $  y $ \sigma_1 = \sigma$.
\item
la \'ultima regla en la derivaci\'on de $\langle t_2, \sigma_2 \rangle$ no pudo ser ASS, SEQ$_1$, SEQ$_2$, IF$_1$ ni IF$_2$ por la estructura de t. Por lo tanto, la \'ultima regla aplicada debi\'o ser tambi\'en REPEAT, entonces\\$t_2 = c;\mathbf{if\ } b \mathbf{\ then\ skip\ else\ repeat\ } c \mathbf{\ until\ } b = t_1$ y $\sigma_2$ = $\sigma$ = $\sigma_1$.
\end{itemize}


\vspace{5mm}


\section{Ejercicio 6}
\paragraph{}
Sean $\sigma_{0} = [\sigma | x:0]$ y $\sigma_{1} = [\sigma | x:1]$. Por definici\'on es claro que $\sigma_{1} = [\sigma_{0} | x:1]$. Sean adem\'as $c_{0} = x:= x+1, b_{0} = x > 0 , c_{1} = x:= x-1, \\c_{2} = \mathbf{\ if\ } b_{0} \mathbf{\ then\ skip\ else\ } c1$.\\Dado que \infer{t\rightsquigarrow^{*} t'}{t \rightsquigarrow t'}, se usar\'a solamente $\rightsquigarrow^{*}$ para acortar la prueba.\\

\subsection{\'Arbol 1}

\infer[\mbox{ASS}]
{\langle x:= x+1, \sigma_{0} \rangle \rightsquigarrow^{*} \langle \mathbf{skip}, \sigma_{1} \rangle}
{\infer[\mbox{PLUS}]
	{\langle x+1, \sigma_{0} \rangle  \Downarrow_{\text{intexp}} 1}
	{\infer[\mbox{VAR}]
		{\langle x, \sigma_{0} \rangle  \Downarrow_{\text{intexp}} 0}
		{}\ \ \ 
	  \infer[\mbox{NVAL}]
		{\langle 1, \sigma_{0}  \rangle \Downarrow_{\text{intexp}} 1}
		{}
	}
}

\vspace{5mm}

\subsection{\'Arbol 2}

\infer[\mbox{TRANSITIVIDAD}]
{\langle c_{0} ; c_{2}, \sigma_{0} \rangle \rightsquigarrow^{*} \langle c_{2}, \sigma_{1} \rangle}
{\infer[\mbox{SEQ}_2]
	{\langle c_{0} ; c_{2}, \sigma_{0} \rangle \rightsquigarrow^{*} \langle \mathbf{\ skip};c_{2}, \sigma_{1} \rangle }
	{\infer[\mbox{ARBOL 1}]
		{\langle x:= x+1, \sigma_{0} \rangle \rightsquigarrow^{*} \langle \mathbf{skip}, \sigma_{1} \rangle}
		{}
	}\ \ \ 
 \infer[\mbox{SEQ}_1]
	{\langle \mathbf{\ skip};c_{2}, \sigma_{1} \rangle \rightsquigarrow^{*} \langle c_{2}, \sigma_{1} \rangle}
	{}
}

\vspace{5mm}

\subsection{\'Arbol 3}

\infer[\mbox{TRANSITIVIDAD}]
{\langle  c_{0} ; c_{2}, \sigma_{0} \rangle \rightsquigarrow^{*} \langle \mathbf{skip}, \sigma_{1} \rangle}
{\infer[\mbox{ARBOL 2}]
	{\langle c_{0} ; c_{2}, \sigma_{0} \rangle \rightsquigarrow^{*} \langle c_{2}, \sigma_{1} \rangle}
	{}\ \ \ 
 \infer[\mbox{IF}_1]
	{\langle c_2, \sigma_{1} \rangle  \rightsquigarrow^{*} \langle \mathbf{skip}, \sigma_{1} \rangle}
	{\infer[\mbox{GT}]
		{\langle b_0, \sigma_{1}  \rangle \Downarrow_{\text{boolexp}} \mathbf{true}}
		{\infer[\mbox{VAR}]
			{\langle x, \sigma_{1} \rangle  \Downarrow_{\text{intexp}} 1}
			{}\ \ \ 
		  \infer[\mbox{NVAL}]
			{\langle 0, \sigma_{1}  \rangle \Downarrow_{\text{intexp}} 0}
			{}
		}
	}
}

\vspace{5mm}

\section{Ejercicio 7}
\lstinputlisting{Eval1.hs}

\section{Ejercicio 8}
\lstinputlisting{Eval2.hs}

\section{Ejercicio 9}
\lstinputlisting{Eval3.hs}


\section{Ejercicio 10}
\subsection{Sintaxis Abstracta}
\begin{align*}
comm ::=\ &\mathbf{skip}\\
|\ &var\ :=\ intexp\\
|\ &comm;comm\\
|\ &\mathbf{if\ } boolexp \mathbf{\ then\  } comm \mathbf{\ else\ } comm\\
|\ &\mathbf{repeat\ }comm\ boolexp\\
|\ &\mathbf{while\ } boolexp \mathbf{\ do\ } comm
\end{align*}

\subsection{Sem\'antica Operacional de Comandos}

\vspace{4mm}

\infer[\mbox{WHILE}_1]
{\langle \mathbf{while\ } b \mathbf{\ do\ } c, \sigma \rangle  \rightsquigarrow  \langle c;\mathbf{while\ } b \mathbf{\ do\ } c, \sigma \rangle }
{\langle b,\sigma \rangle \Downarrow_{\text{boolexp}} \mathbf{true}}

\vspace{4mm}

\infer[\mbox{WHILE}_2]
{\langle \mathbf{while\ } b \mathbf{\ do\ } c, \sigma \rangle  \rightsquigarrow  \langle \mathbf{skip}, \sigma \rangle}
{\langle b,\sigma \rangle \Downarrow_{\text{boolexp}} \mathbf{false}}

\nocite{*}
\bibliography{referencias}
\bibliographystyle{plain}

\end{document}